\chapter{Einleitung}
Das Carl-Zeiss-Gymnasium Jena bietet für mathematisch-naturwissenschaftlich interessierte und begabte Schüler im Raum Ostthüringen verschiedene Korrespondenzzirkel an. Durch anspruchsvolle Aufgaben werden über den Unterricht hinausgehende Fähigkeiten und Fertigkeiten vermittelt.

Bisher wurden die Teilnahmelisten und die von den Schüler erzielten Punkte mit Hilfe von verschiedenen Excel-Tabellen verwaltet. Unser Projekt ersetzt diese durch eine zentrale Datenbank. Die Bedienung erfolgt über ein intuitives Webfrontend. 

Aufgrund der breiten Verfügbarkeit wird das Projekt unter Windows entwickelt. Als Datenbank kommt MySQL zum Einsatz, eine leistungsfähige und kostenfreie Software. Das Webfrontend wird in PHP geschrieben, einer Standardsprache für dynamische Webseiten. Als Entwicklungsumgebung haben wir uns für Eclipse entschieden. Dieses Java-Programm ist Open Source und ebenfalls kostenlos und diente ursprünglich ausschließlich zum Erstellen von Java-Projekten, wurde aber später um weitere Sprachen, wie eben PHP, erweitert. Als Webserver verwenden wir den kostenlosen Apache-Dienst. Diese Kombination (Apache, PHP und MySQL) ist unter Windows als Paket XAMPP verfügbar, kann aber auch leicht unter Linux eingesetzt werden.

Da mehrere Personen am Projekt arbeiten, wird das Quellcodeverwaltungssystem git eingesetzt. Die Daten werden in einem Repository gespeichert und mit github synchronisiert, einem Dienst, der auch kommerzielle Projekte sicher hostet. Um die Bedienung von git zu erleichtern, verwenden wir das Programm TortoiseGit.

Die Dokumentation wird mit LaTex erstellt.
