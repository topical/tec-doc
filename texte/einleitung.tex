\chapter{Einleitung}
Das Carl-Zeiss-Gymnasium Jena bietet für mathematisch-naturwissenschaftlich interessierte und begabte Schüler im Raum Ostthüringen verschiedene Korrespondenzzirkel an. Das Ziel dabei ist es, die optimale Ausprägung und Weiterentwicklung der Schülerpersönlichkeit und ihrer spezifischen Fähigkeiten und  Fertigkeiten über den Unterricht hinaus mit anspruchsvollen Aufgaben, erstellt von Lehrern mit langjähriger Erfahrung. Bisher wurden die Teilnahme und die Auswertung  manuell mit Hilfe von verschiedensten Excel-Tabelle gemacht. Mit unserem Projekt möchten wir das ganze über eine zentrale Datenbank vereinfachen und realisieren. Die Verwaltung soll über eine Website gestaltet werden.\\
\\
Einen Großteil des Projektes wird mit Eclipse erarbeitet. Dieses Java-Programm ist eine kostenlose, open source Entwicklungsumgebung zum Entwickeln von anderen Programmen in Java. Wir nutzen es jedoch für PHP. Das Projekt wird unter Windows entwickelt, da Windows auf den verwendeten Geräten verfügbar ist. (MySQL: leistungsfähiges Datenbanksystem, kostenfrei verfügbar.) Außerdem nutzten wir git. Das ist ein Programm zur Quellcodeverwaltung um gemeinsam am Projekt zuarbeiten. Die Daten werden in einem Repository gespeichert. Dazu kommt die Verwendung von github. Auf einer Seite werden die Repositories abgelegt und kann so mit anderen geteilt werden. Tortoise ist eine Benutzeroberfläche um git über Windows einfach nutzen zu können, was für uns die Arbeit wesentlich erleichtert. Die Dokumentation wird mit LaTex erstellt.
\\
XAMPP
