\chapter{Zusammenfassung}
Das Ziel unseres Informatikprojekts war es, ein Programm zu entwickeln, welches die Verwaltung der Korrespondenzzirkel des Carl-Zeiss-Gymnasiums vereinfacht.
 
Am Anfang hatten wir einige Startschwierigkeiten aufgrund ungenauer, teilweise auch fehlender Aufgabenstellungen und Wünsche, was das Programm können soll. Es kam hinzu, dass wir keine genauen Vorstellungen hatten, wie wir die geforderten Ziele möglichst gut umsetzen können. Ein Hauptproblem war außerdem, dass wir nur wenig über das Programmieren und verschiedene Programme wissen, da wir uns in unserer Freizeit selten damit beschäftigen. Für uns war es ein Herausforderung, die gewünschten Aufgaben umzusetzen.

Wir können jedoch sagen, dass wir am Ende sehr zufrieden mit unserem Ergebnis sind. Das Programm ist in der Lage, die Korrespondenzen, Teilnehmer und Punkte zu verwalten und die Ergebnisse auszuwerten. Es gibt durchaus noch mehr Potential in unserem Projekt, speziell bei den angebotenen Auswertemöglichkeiten. Da alle Daten in einer Datenbank gespeichert sind, ist eine Erweiterung jedoch leicht möglich. Vorerst mussten wir uns aber auf die wichtigen Dinge konzentrieren, um unser Programm zum Laufen zu bringen und haben nebensächliche Sachen erst einmal hinten angestellt. 
