\chapter{Programmablaufprotokolle}

Nachdem man sich angemeldet hat und als Administrator autorisiert ist, kann man über die gesamten Daten verwalten. Somit können auch neue Schulen anlegt werden unter neue Schule hinzufügen. Die neue Schule hat den Namen Musterschule, befindet sich in der Musterstraße 1 in 01234 Musterstadt. Diese kann nun gespeichert werden, daraufhin erscheint die neu angelegte Schule in der Liste. Nun ist es möglich neue Schüler hinzuzufügen, beispielsweise Erika Musterfrau, Max Mustermann und Viktor Vorbeeld. Erika besucht die 9. Klasse, sowie Max und Viktor. Alle 3 gehen auf die Musterschule und wohnen in der Mustergasse in 01234 Musterstadt. Nach dem Speichern werden auch die Schülerdaten in der Datenbank abgelegt. Jetzt ist es möglich ein neues Fach anzulegen, bei dem es möglich ist an Korrespondenzen teilzunehmen, zum Beispiel Biologie. Nun kann man zu den Zirkeln wechseln und einen Zirkel im Fach Biologie Klassenstufe 9 anlegen. Klickt man nun auf Teilnehmer verwalten, werden uns alle Schüler der Klassenstufe 9 angezeigt, also auch unsere Drei aus der Mustergasse. Von diesen drei Schülern wollen aber nur Erika und Viktor am Biologiezirkel teilnehmen, deshalb klickt man bei den beiden auf Eintragen. Sie sind jetzt in dem Zirkel und bekommen Aufgaben zugeschickt. Bei Aufgaben verwalten kann man nun erst einmal die Gesamtpunktzahl für die einzelne Aufgabenserie festlegen und man kann die erreichten Punkte der Schüler eintragen. Erika und Viktor haben 5 Briefe mit Aufgaben zugeschickt bekommen, dabei hat Erika jeweils 4,4,2 und 13 Punkte erhalten, eine Aufgabenserie hatte sie vergessen abzuschicken, dadurch wurden ihr keine Punkte erteilt für den einen Brief. Sie hat insgesamt 27 Punkte, das entspricht 28 \% von der Gesamtpunktzahl von 98 Punkten. Viktor war besser und hat 11,8,19 und 17 Punkte bekommen. Auf eine Aufgabenserie hat er 0 Punkte bekommen, das heißt nicht, dass er nichts abgeschickt hat, sondern er hat einfach keine richtigen Lösungen gehabt. Im Unterschied zu Erika bleibt das Feld bei der Aufgabenserie nicht frei, sondern es stehen 0 Punkte darin. Viktor hat insgesamt 55 Punkte bekommen, also rund 56 \%. Außerdem kann man nun unter der Musterschule sehen, welche die aktiven Teilnehmer sind. Es sind nur Erika und Viktor, da sie tatsächlich an einer Korrespondenz teilnehmen, im Gegensatz zu Max.