
% Glossar

\chapter*{Glossar}

\begin{description}
\item [Framework] ist ein Programmiergerüst, um das Erstellen von Programmen zu vereinfachen

\item [git] ist ein Quellcodeverwaltungsystem

\item [Laravel] ist Framework, um in PHP Web-Anwendungen zu erstellen

\item [Java] ist eine Programmiersprache

\item [JavaScript] ist eine Skriptsprache, um direkt im Browser Programme auszuführen
	
\item [MySQL] ist ein Datenbanksystem

\item [PHP] ist eine Programmiersprache zur Erstellung von dynamischen Webseiten 

\item [Repository] ist bei der Quellcodeverwaltung ein Verzeichnis, in dem alle Dateien und Versionsinformationen eines Projekts gespeichert sind

\item [Web-Frontend] ist die grafische Benutzeroberfläche einer datenbankbasierten Web-Anwendung; das Gegenstück ist die Datenbank, genannt Backend

\item [XAMPP] ist ein Programmpaket für Windows, bestehend aus Apache, MySQL, Perl und PHP 

\end{description}
